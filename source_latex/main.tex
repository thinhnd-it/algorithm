\documentclass[a4paper]{article}
\usepackage{vntex}
%\usepackage[english,vietnam]{babel}
%\usepackage[utf8]{inputenc}
\usepackage{csvsimple}
% new pakage
\usepackage{framed}
% end new pakage

%\usepackage[utf8]{inputenc}
%\usepackage[francais]{babel}
\usepackage{a4wide,amssymb,epsfig,latexsym,multicol,array,hhline,fancyhdr}
\usepackage{booktabs}
\usepackage{amsmath}
\usepackage{lastpage}
\usepackage[lined,boxed,commentsnumbered]{algorithm2e}
\usepackage{enumerate}
\usepackage{color}
\usepackage{graphicx}							% Standard graphics package
\usepackage{array}
\usepackage{tabularx, caption}
\usepackage{multirow}
\usepackage[framemethod=tikz]{mdframed}% For highlighting paragraph backgrounds
\usepackage{multicol}
\usepackage{rotating}
\usepackage{graphics}
\usepackage{geometry}
\usepackage{setspace}
\usepackage{epsfig}
\usepackage{tikz}
\usepackage{listings}
\usepackage{xcolor}
\usetikzlibrary{arrows,snakes,backgrounds}
\usepackage{hyperref}
\hypersetup{urlcolor=blue,linkcolor=black,citecolor=black,colorlinks=true} 
%\usepackage{pstcol} 								% PSTricks with the standard color package

\newtheorem{theorem}{{\bf Định lý}}
\newtheorem{property}{{\bf Tính chất}}
\newtheorem{proposition}{{\bf Mệnh đề}}
\newtheorem{corollary}[proposition]{{\bf Hệ quả}}
\newtheorem{lemma}[proposition]{{\bf Bổ đề}}

% \everymath{\color{blue}}
%\usepackage{fancyhdr}
\setlength{\headheight}{40pt}
\pagestyle{fancy}
\fancyhead{} % clear all header fields
\fancyhead[L]{
 \begin{tabular}{rl}
    \begin{picture}(25,15)(0,0)
    \put(0,-8){\includegraphics[width=8mm, height=8mm]{uit.jpg}}
    %\put(0,-8){\epsfig{width=10mm,figure=hcmut.eps}}
   \end{picture}&
	%\includegraphics[width=8mm, height=8mm]{hcmut.png} & %
	\begin{tabular}{l}
		\textbf{\bf \ttfamily TRƯỜNG ĐẠI HỌC CÔNG NGHỆ THÔNG TIN}\\
		\textbf{\bf \ttfamily KHOA KHOA HỌC MÁY TÍNH}
	\end{tabular} 	
 \end{tabular}
}
\fancyhead[R]{
	\begin{tabular}{l}
		\tiny \bf \\
		\tiny \bf 
	\end{tabular}  }
\fancyfoot{} % clear all footer fields
\fancyfoot[R]{\scriptsize \ttfamily Trang {\thepage}/\pageref{LastPage}}
\renewcommand{\headrulewidth}{0.3pt}
\renewcommand{\footrulewidth}{0.3pt}


%%%
\setcounter{secnumdepth}{4}
\setcounter{tocdepth}{3}
\makeatletter
\newcounter {subsubsubsection}[subsubsection]
\renewcommand\thesubsubsubsection{\thesubsubsection .\@alph\c@subsubsubsection}
\newcommand\subsubsubsection{\@startsection{subsubsubsection}{4}{\z@}%
                                     {-3.25ex\@plus -1ex \@minus -.2ex}%
                                     {1.5ex \@plus .2ex}%
                                     {\normalfont\normalsize\bfseries}}
\newcommand*\l@subsubsubsection{\@dottedtocline{3}{10.0em}{4.1em}}
\newcommand*{\subsubsubsectionmark}[1]{}
\makeatother

\definecolor{dkgreen}{rgb}{0,0.6,0}
\definecolor{gray}{rgb}{0.5,0.5,0.5}
\definecolor{mauve}{rgb}{0.58,0,0.82}
\definecolor{codegreen}{rgb}{0,0.6,0}
\definecolor{codegray}{rgb}{0.5,0.5,0.5}
\definecolor{codepurple}{rgb}{0.58,0,0.82}
\definecolor{backcolour}{rgb}{0.95,0.95,0.92}
\newcommand\tab[1][1cm]{\hspace*{#1}}

\lstdefinestyle{mystyle}{
  backgroundcolor=\color{backcolour},   commentstyle=\color{codegreen},
  keywordstyle=\color{magenta},
  numberstyle=\tiny\color{codegray},
  stringstyle=\color{codepurple},
  basicstyle=\ttfamily\footnotesize,
  breakatwhitespace=false,         
  breaklines=true,                 
  captionpos=b,                    
  keepspaces=true,                 
  numbers=left,                    
  numbersep=5pt,                  
  showspaces=false,                
  showstringspaces=false,
  showtabs=false,                  
  tabsize=2
}

\lstset{frame=tb,
	language=Matlab,
	aboveskip=3mm,
	belowskip=3mm,
	showstringspaces=false,
	columns=flexible,
	basicstyle={\small\ttfamily},
	numbers=none,
	numberstyle=\tiny\color{gray},
	keywordstyle=\color{blue},
	commentstyle=\color{dkgreen},
	stringstyle=\color{mauve},
	breaklines=true,
	breakatwhitespace=true,
	tabsize=3,
	numbers=left,
	stepnumber=1,
	numbersep=1pt,    
	firstnumber=1,
	numberfirstline=true
}
\lstset{style=mystyle}
\begin{document}

\begin{titlepage}
\begin{center}
ĐẠI HỌC QUỐC GIA THÀNH PHỐ HỒ CHÍ MINH \\
TRƯỜNG ĐẠI HỌC CÔNG NGHỆ THÔNG TIN \\
KHOA KHOA HỌC MÁY TÍNH 
\end{center}

\vspace{1cm}

\begin{figure}[h!]
\begin{center}
\includegraphics[width=5cm]{uit.jpg}
\end{center}
\end{figure}

\vspace{1cm}


\begin{center}
\begin{tabular}{c}
	{\textbf{{\Large PHÂN TÍCH THIẾT KẾ THUẬT TOÁN}}}\\
	~~\\
	\hline
	\\
	\textbf{{\Huge Bài Toán Subset Sum

}}\\
	\\
	\hline
\end{tabular}
\end{center}

\vspace{2cm}

\begin{table}[h]
\begin{tabular}{rrl}
\hspace{5 cm}
\medskip
& GV: & Phạm Nguyễn Trường An\\
& SV: & Nguyễn Đức Thịnh - 18521442\\
& & Huỳnh Minh Tuấn - 18521596 \\
& & Nguyễn Minh Thông - 18521459 \\
& & Phan Phát Huy - 18520287\\
\end{tabular}
\end{table}

\end{titlepage}


\thispagestyle{empty}

\newpage
\tableofcontents
\newpage
\section{Giới Thiệu Bài Toán }
\subsection{Vấn đề}
Bài toán Subset Sum là bài toán tìm kiếm trong tập hợp cho trước một tập hợp con có tổng bằng tổng cho trước.\medskip \\
Subset Sum được biết đến là bài toán NP-complete(Một bài toán L là NP-complete nếu nó nằm trong lớp NP (lời giải cho L có thể được kiểm chứng trong thời gian đa thức)).
\subsection{Mô tả hình thức}
\textbf{Input:} Set $A$ có $n$ phần tử là số nguyên dương và tổng cho trước $Sum$ là một số nguyên không âm. \medskip \\
\textbf{Output:} Nếu tồn tại ít nhất một tập hợp con có tổng bằng $Sum$ thì xuất ra $True$, ngược lại xuất ra $False$. Case nâng cao xuất ra sum maximum có thể được tạo ra nhưng không vượt quá $Sum$
\subsection{Lịch sử}
Là một vấn đề đã xuất hiện từ rất lâu trong quá trình nghiên cứu và phát triển tư duy của loài người trong lĩnh vực toán, tin học. Trong quá trình đó nhiều đề xuất để giải quyết vấn đề này đã được đề xuất như backtracking, bruteforce, và điển hình là dynamic programing được Bell Man đề xuất năm 1957.
\subsection{Ứng dụng:}
Có rất nhiều ứng dụng của Subset Sum trong đời sống, điển hình như Computer Password, Message Verification,...
\subsubsection{Computer Password}
Máy tính cần xác minh danh tính của người dùng trước khi cho phép họ truy cập vào tài khoản.  Hệ thống đơn giản nhất sẽ yêu cầu máy giữ một bản sao của mật khẩu trong một tệp nội bộ và so sánh nó với những gì người dùng nhập.  Một hạn chế là bất kỳ ai nhìn thấy tệp nội bộ sau đó đều có thể mạo danh người dùng. Máy tính trước tiên sẽ sản sinh ra 1 tập lớn các phần tử $ai$ (chẳng hạn 500) và lưu trong local storage. Password là subset của {1,..500}. Thay vì có mật khẩu cho người dùng, máy tính sẽ lưu tổng của subset. Khi người dùng nhập subset máy tính sẽ kiểm tra xem tổng có đúng hay không. Máy tính không lưu lại bản ghi nào của subset. Thế nên việc mạo danh chỉ xảy ra khi người khác biết tổng và subset. 

\subsection{Ứng dụng thực tế}
Lazada là một nền tảng thương mại điện tử có số má ở Việt Nam. Vào mỗi dịp lễ Tết Lazada sẽ có những đợt giảm giá nếu mua đủ một khoảng $n$ đồng nào đó. Tuy nhiêu chương trình chỉ cho phép mua mỗi mặt hàng với đơn vị là 1. Hãy viết chương trình giúp người dùng có thể biết được mình có húp được khuyến mãi này không. \medskip \\
\textbf{Input:} Tập hợp $A$ với $n$ phần tử thể hiện là số lượng hàng mà Lazada có trong mùa khuyến mãi mỗi phần tử có key là tên sản phẩm, value là giá tiền của sản phẩm đó, và giá tiền $Sum$ phải đạt để húp khuyến mãi.  \medskip \\
\textbf{Output:} Nếu tồn tại ít nhất một tập hợp con có tổng bằng $Sum$ thì thông báo khách hàng có thể đạt được khuyến mãi. Có thể trace back để tìm đó là những sản phẩm nào. Nếu không thể đạt được $Sum$ thì báo cho khách hàng biết nên từ bỏ Lazada và chuyển sang nền tảng Shopee vì bên đó không có chơi khách hàng kiểu này. 
\newpage

\section{Các Ý Tưởng Thiết Kế Thuật Toán:}
\subsection{Phát sinh test case}
Phát sinh test case là một bước tất yếu của quá trình phát triển phân tích thiết kế một thuật toán. \\
Thay vì phải đau đầu nghĩ ra một test case nào đó cho bài toán đã có sẵn và đã được phân tích mòn từ nhiều thập kỷ qua nhóm nghĩ ngay tới việc sử dụng các test case có sẵn của các trang pratice code như hacker rank, leetcode, ... \\
Tuy nhiên các trang này thường không công bố test case giống như trang Wecode. Vậy nên nhóm sự dụng những test case có sẵn được công bố trong link sau \url{https://people.sc.fsu.edu/~jburkardt/datasets/subset_sum/subset_sum.html}.\\ Nếu thuật toán pass tất cả các test thì chứng tỏ được tính đúng đắn của nó \\
\subsection{Phương pháp quay lui }
\subsubsection{Khái quát phương pháp}
Backtraking là một kĩ thuật thiết kế giải thuật dựa trên đệ quy. Ý tưởng của
Backtraking là tìm lời giải từng bước, mỗi bước chọn một trong số các lựa chọn
khả dĩ và đệ quy. Người đầu tiên đề ra thuật ngữ này (backtrack) là nhà toán học
người Mỹ D. H. Lehmer vào những năm 1950.\\ \\
\textit{Ưu điểm: }\\
Việc quay lui là thử tất cả các tổ hợp để tìm được một lời giải. Thế mạnh của
phương pháp này là nhiều cài đặt tránh được việc phải thử nhiều trường hợp chưa
hoàn chỉnh, nhờ đó giảm thời gian chạy\\ \\
\textit{Nhược điểm:}\\
Trong trường hợp xấu nhất độ phức tạp của quay lui vẫn là cấp số mũ. Vì nó mắc
phải các nhược điểm sau:\\
\tab - Rơi vào tình trạng "thrashing": qúa trình tìm kiếm cứ gặp phải bế tắc với
cùng một nguyên nhân.\\
\tab - Thực hiện các công việc dư thừa: Mỗi lần chúng ta quay lui, chúng ta cần
phải đánh giá lại lời giải trong khi đôi lúc điều đó không cần thiết.\\
\tab - Không sớm phát hiện được các khả năng bị bế tắc trong tương lai. Quay lui
chuẩn, không có cơ chế nhìn về tương lai để nhận biết đc nhánh tìm kiếm sẽ đi vào
bế tắc.
\subsubsection{Mã giả}
\newpage
\begin{framed}
\textbf{SubsetSum}(X(1,2,....n), n ,T):\\
\tab \textbf{Input:} T là tổng muốn đạt được\\
\tab \textbf{Input:} X là tập hợp được cho \\
\tab \textbf{Input:} n là Số phần tử của tập hợp X\\
\tab if T = 0 : \\
\tab\tab return True \\
\tab else if T < 0 or n == 0 :\\
\tab\tab return False\\
\tab return \textbf{SubsetSum}(X[1,2,...,n-1],n-1, T) or \textbf{SubsetSum}(X[1,2,...,n-1],n-1,T-X[n])
\end{framed}

\subsubsection{Phân tích độ phức tạp bằng các phương pháp toán học}
\tab \textbf{Phương trình đệ quy:}
\[T(n) = \begin{cases} C_1, & n=0 \\ 2T(n-1), & n>0 \end{cases}\]
\\
Giải phương trình đệ quy:
$$T(n) = 2T(n-1)$$
$$\Leftrightarrow T(n) = 4T(n-2)$$
$$\Leftrightarrow T(n) = 8T(n-3)$$
$$\Leftrightarrow T(n) = 16T(n-4)$$\\
Sau i bước thay thế ta được:
$$T(n) = 2^i T(n-i)$$\\
Khi $n=i$:
$$T(n) = 2^n T(0)$$
$$\Leftrightarrow T(n) = 2^n C_1$$\\
Vậy độ phức tạp của thuật toán là : $ O(2^n)$


\subsubsection{Mã nguồn :}
\begin{center}
    \begin{lstlisting}[language=Python, caption=Solve by Backtracking]
def isSubsetSum(set, n, sum):
 
    # Base Cases
    if (sum == 0):
        return True
    if (n == 0):
        return False
 
    # If last element is greater than
    # sum, then ignore it
    if (set[n - 1] > sum):
        return isSubsetSum(set, n - 1, sum)
 
    # else, check if sum can be obtained
    # by any of the following
    # (a) including the last element
    # (b) excluding the last element
    return isSubsetSum(
        set, n-1, sum) or isSubsetSum(
        set, n-1, sum-set[n-1])
\end{lstlisting}
\end{center}

\subsubsection{Phân tích độ phức tạp cài đặt bằng thực nghiệm}
\subsubsubsection{Cấu hình máy thực nghiệm}
\begin{itemize}
  \item Chip: Intel Core i5 8250U
  \item RAM: 4GB
  \item Không card đồ hoạ
\end{itemize}
\subsubsubsection{Cách thức thực nghiệm}
Ở mỗi lần gọi hàm \textit{isSubsetSum} ta đều random 1 tập hợp $set$ với số phần tử n tăng dần từ 20 đến 89. Đặt mốc thời gian trước và sau khi gọi hàm \textit{isSubsetSum} để tính time diff và ghi kết quả vào cột $tn$ tương ứng với mỗi $n$, quy trình này được thực hiện tự động bằng code.\\ \\
Tiếp tục tính $y = f(n)$ với $f(n)$ là các hàm độ phức tạp thông dụng đã biết. \\ \\
Sử dụng linear regrestion trên các giá trị tính được để tính hệ số X và MSE.\\ \\
Ta có kết quả thực nghiệm như bên dưới.
\subsubsubsection{Kết quả thực nghiệm}
\newpage
\scalebox{0.7}
{\begin{tabular}{|l|r|r|r|r|r|r|r|r|r|}
    \hline
    n     & {tn} & {46.9693*lgn} & {20.6208*sqrtn} & {1.4961*n} & {0.211087*nlgn} & {0.0139272*n\^2} & {0.00015241*n\^3} & {1.40885E-25*2\^n} & {3.47841E-135*n!} \\
    {20} & 0.181176 & 202.9979 & 92.21902 & 29.922 & 18.24606 & 5.57088 & 1.21928 & 1.48E-19 & 8.45E-117 \\
    {21} & 0.197245 & 206.3041 & 94.49638 & 31.4181 & 19.47038 & 6.141895 & 1.411469 & 2.95E-19 & 1.78E-115 \\
    {22} & 0.227558 & 209.4564 & 96.72013 & 32.9142 & 20.70922 & 6.740765 & 1.622862 & 5.91E-19 & 3.90E-114 \\
    {23} & 0.277083 & 212.4685 & 98.89388 & 34.4103 & 21.9619 & 7.367489 & 1.854372 & 1.18E-18 & 9.01E-113 \\
    {24} & 0.311321 & 215.3525 & 101.0209 & 35.9064 & 23.22782 & 8.022067 & 2.106916 & 2.36E-18 & 2.16E-111 \\
    {25} & 0.385314 & 218.1187 & 103.104 & 37.4025 & 24.50644 & 8.7045 & 2.381406 & 4.73E-18 & 5.39E-110 \\
    {26} & 0.522199 & 220.7764 & 105.1459 & 38.8986 & 25.79724 & 9.414787 & 2.678758 & 9.45E-18 & 1.40E-108 \\
    {27} & 0.632704 & 223.3337 & 107.1488 & 40.3947 & 27.09976 & 10.15293 & 2.999886 & 1.89E-17 & 3.79E-107 \\
    {28} & 0.682402 & 225.7981 & 109.115 & 41.8908 & 28.41356 & 10.91892 & 3.345704 & 3.78E-17 & 1.06E-105 \\
    {29} & 1.190402 & 228.176 & 111.0464 & 43.3869 & 29.73824 & 11.71278 & 3.717127 & 7.56E-17 & 3.07E-104 \\
    {30} & 0.995859 & 230.4732 & 112.9448 & 44.883 & 31.07342 & 12.53448 & 4.11507 & 1.51E-16 & 9.22E-103 \\
    {31} & 1.136592 & 232.6951 & 114.8118 & 46.3791 & 32.41876 & 13.38404 & 4.540446 & 3.03E-16 & 2.86E-101 \\
    {32} & 1.279592 & 234.8465 & 116.6489 & 47.8752 & 33.77392 & 14.26145 & 4.994171 & 6.05E-16 & 9.15E-100 \\
    {33} & 2.732696 & 236.9317 & 118.4575 & 49.3713 & 35.1386 & 15.16672 & 5.477158 & 1.21E-15 & 3.02E-98 \\
    {34} & 1.997775 & 238.9546 & 120.2389 & 50.8674 & 36.51251 & 16.09984 & 5.990323 & 2.42E-15 & 1.03E-96 \\
    {35} & 3.770094 & 240.9188 & 121.9943 & 52.3635 & 37.89537 & 17.06082 & 6.534579 & 4.84E-15 & 3.58E-95 \\
    {36} & 2.65801 & 242.8278 & 123.7248 & 53.8596 & 39.28694 & 18.04965 & 7.110841 & 9.68E-15 & 1.29E-93 \\
    {37} & 2.804384 & 244.6844 & 125.4314 & 55.3557 & 40.68697 & 19.06634 & 7.720024 & 1.93E-14 & 4.80E-92 \\
    {38} & 3.465906 & 246.4915 & 127.1151 & 56.8518 & 42.09523 & 20.11088 & 8.363042 & 3.87E-14 & 1.82E-90 \\
    {39} & 4.227073 & 248.2516 & 128.7769 & 58.3479 & 43.51151 & 21.18327 & 9.040809 & 7.75E-14 & 7.10E-89 \\
    {40} & 4.564096 & 249.9672 & 130.4174 & 59.844 & 44.93559 & 22.28352 & 9.75424 & 1.55E-13 & 2.84E-87 \\
    {41} & 5.584631 & 251.6405 & 132.0375 & 61.3401 & 46.36729 & 23.41162 & 10.50425 & 3.10E-13 & 1.17E-85 \\
    {42} & 5.436054 & 253.2734 & 133.6381 & 62.8362 & 47.80642 & 24.56758 & 11.29175 & 6.20E-13 & 4.90E-84 \\
    {43} & 6.539495 & 254.8679 & 135.2196 & 64.3323 & 49.2528 & 25.75139 & 12.11766 & 1.24E-12 & 2.10E-82 \\
    {44} & 8.106527 & 256.4257 & 136.7829 & 65.8284 & 50.70626 & 26.96306 & 12.98289 & 2.48E-12 & 9.25E-81 \\
    {45} & 8.26735 & 257.9485 & 138.3285 & 67.3245 & 52.16665 & 28.20258 & 13.88836 & 4.96E-12 & 4.17E-79 \\
    {46} & 11.90808 & 259.4378 & 139.8571 & 68.8206 & 53.6338 & 29.46996 & 14.83498 & 9.92E-12 & 1.91E-77 \\
    {47} & 9.980406 & 260.8952 & 141.3691 & 70.3167 & 55.10757 & 30.76518 & 15.82366 & 1.99E-11 & 9.01E-76 \\
    {48} & 12.76195 & 262.3218 & 142.8651 & 71.8128 & 56.58782 & 32.08827 & 16.85533 & 3.96E-11 & 4.31E-74 \\
    {49} & 13.26711 & 263.719 & 144.3456 & 73.3089 & 58.07442 & 33.43921 & 17.93088 & 7.93E-11 & 2.11E-72 \\
    {50} & 11.83909 & 265.088 & 145.8111 & 74.805 & 59.56723 & 34.818 & 19.05125 & 1.59E-10 & 1.06E-70 \\
    {51} & 18.03763 & 266.4298 & 147.262 & 76.3011 & 61.06614 & 36.22465 & 20.21734 & 3.17E-10 & 5.39E-69 \\
    {52} & 16.08689 & 267.7457 & 148.6987 & 77.7972 & 62.57101 & 37.65915 & 21.43007 & 6.34E-10 & 2.81E-67 \\
    {53} & 18.12508 & 269.0364 & 150.1217 & 79.2933 & 64.08175 & 39.1215 & 22.69034 & 1.27E-09 & 1.49E-65 \\
    {54} & 22.18074 & 270.303 & 151.5313 & 80.7894 & 65.59822 & 40.61172 & 23.99909 & 2.54E-09 & 8.04E-64 \\
    {55} & 21.47587 & 271.5464 & 152.9279 & 82.2855 & 67.12034 & 42.12978 & 25.35721 & 5.07E-09 & 4.42E-62 \\
    {56} & 25.04538 & 272.7674 & 154.3119 & 83.7816 & 68.648 & 43.6757 & 26.76563 & 1.02E-08 & 2.47E-60 \\
    {57} & 29.88423 & 273.9668 & 155.6836 & 85.2777 & 70.18109 & 45.24947 & 28.22527 & 2.03E-08 & 1.41E-58 \\
    {58} & 28.4651 & 275.1453 & 157.0433 & 86.7738 & 71.71953 & 46.8511 & 29.73702 & 4.06E-08 & 8.17E-57 \\
    {59} & 32.13041 & 276.3036 & 158.3914 & 88.2699 & 73.26322 & 48.48058 & 31.30181 & 8.11E-08 & 4.83E-55 \\
    {60} & 32.11618 & 277.4425 & 159.728 & 89.766 & 74.81207 & 50.13792 & 32.92056 & 1.62E-07 & 2.89E-53 \\
    {61} & 31.89916 & 278.5626 & 161.0536 & 91.2621 & 76.36599 & 51.82311 & 34.59417 & 3.25E-07 & 1.77E-51 \\
    {62} & 41.21547 & 279.6644 & 162.3683 & 92.7582 & 77.92491 & 53.53616 & 36.32357 & 6.49E-07 & 1.10E-49 \\
    {63} & 38.83752 & 280.7487 & 163.6725 & 94.2543 & 79.48874 & 55.27706 & 38.10966 & 1.30E-06 & 6.89E-48 \\
    {64} & 42.40097 & 281.8158 & 164.9664 & 95.7504 & 81.05741 & 57.04581 & 39.95337 & 2.59E-06 & 4.42E-46 \\
    {65} & 44.48521 & 282.8664 & 166.2502 & 97.2465 & 82.63083 & 58.84242 & 41.8556 & 5.20E-06 & 2.87E-44 \\
    {66} & 52.45659 & 283.901 & 167.5242 & 98.7426 & 84.20894 & 60.66688 & 43.81727 & 1.04E-05 & 1.89E-42 \\
    {67} & 58.10673 & 284.92 & 168.7885 & 100.2387 & 85.79166 & 62.5192 & 45.83929 & 2.09E-05 & 1.27E-40 \\
    {68} & 52.98158 & 285.9239 & 170.0435 & 101.7348 & 87.37893 & 64.39937 & 47.92258 & 4.16E-05 & 8.63E-39 \\
    {69} & 53.47965 & 286.9131 & 171.2892 & 103.2309 & 88.97068 & 66.3074 & 50.06806 & 8.31E-05 & 5.95E-37 \\
    {70} & 52.46033 & 287.8881 & 172.526 & 104.727 & 90.56684 & 68.24328 & 52.27663 & 1.66E-04 & 4.17E-35 \\
    {71} & 52.25237 & 288.8493 & 173.7539 & 106.2231 & 92.16735 & 70.20702 & 54.54922 & 3.32E-04 & 2.96E-33 \\
    {72} & 55.56277 & 289.7971 & 174.9733 & 107.7192 & 93.77215 & 72.1986 & 56.88673 & 6.65E-04 & 2.13E-31 \\
    {73} & 66.38579 & 290.7317 & 176.1842 & 109.2153 & 95.38118 & 74.21805 & 59.29008 & 1.33E-03 & 1.55E-29 \\
    {74} & 65.29081 & 291.6537 & 177.3868 & 110.7114 & 96.99438 & 76.26535 & 61.76019 & 2.66E-03 & 1.15E-27 \\
    {75} & 65.84824 & 292.5633 & 178.5814 & 112.2075 & 98.6117 & 78.3405 & 64.29797 & 5.33E-03 & 8.63E-26 \\
    {76} & 76.25178 & 293.4608 & 179.768 & 113.7036 & 100.2331 & 80.44351 & 66.90433 & 1.07E-02 & 6.57E-24 \\
    {77} & 67.78732 & 294.3466 & 180.9468 & 115.1997 & 101.8585 & 82.57437 & 69.58019 & 2.13E-02 & 5.04E-22 \\
    {78} & 69.78137 & 295.2209 & 182.118 & 116.6958 & 103.4878 & 84.73308 & 72.32647 & 4.25E-02 & 3.93E-20 \\
    {79} & 76.54159 & 296.0842 & 183.2817 & 118.1919 & 105.121 & 86.91966 & 75.14407 & 8.51E-02 & 3.11E-18 \\
    {80} & 80.96433 & 296.9365 & 184.438 & 119.688 & 106.7581 & 89.13408 & 78.03392 & 1.70E-01 & 2.49E-16 \\
    {81} & 87.53794 & 297.7783 & 185.5872 & 121.1841 & 108.3991 & 91.37636 & 80.99692 & 3.41E-01 & 2.02E-14 \\
    {82} & 92.25078 & 298.6098 & 186.7293 & 122.6802 & 110.0437 & 93.64649 & 84.034 & 6.82E-01 & 1.65E-12 \\
    {83} & 78.86256 & 299.4311 & 187.8644 & 124.1763 & 111.6921 & 95.94448 & 87.14606 & 1.36E+00 & 1.37E-10 \\
    {84} & 86.03089 & 300.2427 & 188.9928 & 125.6724 & 113.3441 & 98.27032 & 90.33402 & 2.72E+00 & 1.15E-08 \\
    {85} & 90.4318 & 301.0446 & 190.1144 & 127.1685 & 114.9998 & 100.624 & 93.59879 & 5.45E+00 & 9.81E-07 \\
    {86} & 88.90865 & 301.8372 & 191.2294 & 128.6646 & 116.6591 & 103.0056 & 96.94129 & 1.09E+01 & 8.42E-05 \\
    {87} & 82.86155 & 302.6205 & 192.338 & 130.1607 & 118.3219 & 105.415 & 100.3624 & 2.18E+01 & 7.34E-03 \\
    {88} & 91.00732 & 303.395 & 193.4403 & 131.6568 & 119.9882 & 107.8522 & 103.8631 & 4.35E+01 & 6.44E-01 \\
    {89} & 88.80662 & 304.1607 & 194.5362 & 133.1529 & 121.6579 & 110.3174 & 107.4443 & 8.72E+01 & 5.74E+01 \\
    MSE   &       & 54321.7 & 13696.9 & 2444.83 & 1263.8 & 226.597 & 30.8483 & 1.78E+03 & 1.98E+03 \\
    \hline
\end{tabular}}
\newpage
Từ thực nghiệm nhận thấy độ phức tạp của thuật toán là : $ O(n) = n^3$\\ \\
Tuy đã thực nghiệm lại nhưng vẫn không thu được kết quả $2^n$
\subsection{Phương pháp quy hoạch động }
\subsubsection{Khái quát phương pháp}
Quy hoạch động là kĩ thuật được dùng khi có một công thức hoặc một (một vài)
trạng thái bắt đầu. Một bài toán được tính bởi các bài toán nhỏ hơn đã được tìm
ra từ trước, và kết quả các bài toán sẽ được lưu lại để những lần tính toán tiếp
theo nếu cần đến những kết quả đó thì không cần tốn thêm thời gian thực hiện lại những bài toán này nữa. Nói cách khác, quy hoạch động bắt đầu từ việc giải các
bài toán nhỏ, để từ đó từng bước giải quyết bài toán lớn hơn , và cuối cùng là bài
toán lớn nhất ( bài toán ban đầu).\\ \\

\textit{Ý tưởng:} Sử dụng hướng tiếp cận\textbf{ bottom - up} để xây dựng mảng 2 chiều, ta tối ưu sum từ 1 → n với các giá trị trong set.\\ \\
\textit{Ưu điểm: }\\
\tab - Chương trình chạy nhanh,
mang tính tối ưu hóa cao.\\ \\
\textit{Nhược điểm:}\\
\tab - Sự kết hợp giữa các bài toán con chưa chắc cho ta bài toán lớn.\\
\tab - Số lượng các bài toán con cần giải quyết có thể rất lớn.

\subsubsection{Mã giả}

\begin{framed}
public $function$ \textbf{SubsetSum}(Set, n ,Sum):\\
\tab \textbf{Input: } Set là tập hợp cho trước\\
\tab \textbf{Input: } n là số lượng phần tử trong Set\\
\tab \textbf{Input: } Sum là Tổng cho trước\\
\tab Đặt Subset là mảng 2 chiều (n+1)(Sum+1) lưu các giá trị $False$\\
\tab //nếu Sum = 0 thì kết quả là $True$  \\ \\
\tab for i = 1 to n + 1 do:\\
\tab\tab Subset[i][0] = $True$\\ \\
\tab // nếu Sum != 0 và Set = [] thì kết quả là $False$ \\ \\ 
\tab for i = 2 to Sum + 1 do: \\
\tab\tab Subset[0][i] = $False$ \\ \\
\tab for i = 2 to n + 1 do : \\
\tab \tab for j = 2  to sum + 1 do:\\
\tab \tab \tab if Set[i] > j then : \\
\tab \tab \tab \tab Subset[i][j] = Subset[i-1][j] \\
\tab \tab \tab else: \\
\tab \tab \tab \tab Subset[i][j] = Subset[i-1][j] or Subset[i-1][j-set[i]\\ \\
\tab return Subset[n][Sum]


\tab\tab  \\
\tab 
\end{framed}

\tab \textbf{Độ phức tạp của thuật toán là : $O(n) = n*sum$} trong đó sum là tổng muốn tìm \\ \tab subset và n là số phần tử của tập hợp\\\\
\tab \textbf{Độ phức tạp bộ nhớ là: $O(n) = n*sum$ } trong đó $n*sum$ là kích thước của ma \\ \tab trận  2D
.\subsubsection{Mã nguồn}
\begin{center}
    \begin{lstlisting}[language=Python, caption=Solve Dynamic Programing]
def isSubsetSum(set, n, sum):
     
    # The value of subset[i][j] will be 
    # true if there is a
    # subset of set[0..j-1] with sum equal to i
    subset =([[False for i in range(sum + 1)] 
            for i in range(n + 1)])
     
    # If sum is 0, then answer is true 
    for i in range(n + 1):
        subset[i][0] = True
         
    # If sum is not 0 and set is empty, 
    # then answer is false 
    for i in range(1, sum + 1):
         subset[0][i]= False
             
    # Fill the subset table in botton up manner
    for i in range(1, n + 1):
        for j in range(1, sum + 1):
            if j<set[i-1]:
                subset[i][j] = subset[i-1][j]
            if j>= set[i-1]:
                subset[i][j] = (subset[i-1][j] or
                                subset[i - 1][j-set[i-1]])
     
    return subset[n][sum]
\end{lstlisting}
\end{center}
\subsubsection{Trace Back}
\subsubsubsection{Mã giả}
\begin{framed}
\textbf{traceBack}(DP, n, sum, set):\\
\tab \textbf{Input:} DP là matran dynamic programing nếu hàm isSubsetSum return về\\
\tab \textbf{Input:} set là tập hợp được cho \\
\tab \textbf{Input:} n là Số phần tử của tập hợp set\\
\tab \textbf{Input:} sum là tổng muốn đạt được\\
\tab Đặt m = n, b = sum \\
\tab Khởi tạo mảng subset rỗng để lưu lại path \\
\tab while b > 0 : \\
\tab\tab if DP[m-1][b] == true do \\
\tab\tab\tab m = m - 1 \\
\tab\tab else :\\
\tab\tab\tab m = m - 1 \\
\tab\tab\tab push set[m] into subset \\
\tab\tab\tab b = b - set[m] \\
\tab print subset
\end{framed}
\subsubsubsection{Mã nguồn}
\begin{center}
    \begin{lstlisting}[language=Python, caption=Trace back Dynamic Programing]
def traceBack(DP, sum, n, set):
    m, b = n, sum
    subset = []
    while b > 0:
        if DP[m-1][b]:
            m -= 1
        else:
            m -= 1
            subset.append(set[m])
            b -= set[m]
    print(subset)
\end{lstlisting}
\end{center}
\subsubsection{Phân tích độ phức tạp bằng phương pháp thực nghiệm}
\subsubsubsection{Cấu hình máy thực nghiệm}
\begin{itemize}
  \item Chip: Intel Core i5 8250U
  \item RAM: 4GB
  \item Không card đồ hoạ
\end{itemize}
\subsubsubsection{Cách thức thực nghiệm}
Ở mỗi lần gọi hàm \textit{isSubsetSum} ta đều random 1 tập hợp $set$ với số phần tử n tăng dần từ 20 đến 59. Đặt mốc thời gian trước và sau khi gọi hàm \textit{isSubsetSum} để tính time diff và ghi kết quả vào cột $tn$ tương ứng với mỗi $n$, quy trình này được thực hiện tự động bằng code.\\ \\
Tiếp tục tính $y = f(n)$ với $f(n)$ là các hàm độ phức tạp thông dụng đã biết. \\ \\
Sử dụng linear regrestion trên các giá trị tính được để tính hệ số X và MSE.\\ \\
Ta có kết quả thực nghiệm như bên dưới.
\subsubsubsection{Kết quả thực nghiệm}

\scalebox{0.58}{
\begin{tabular}{|l|r|r|r|r|r|r|r|r|r|}
    \hline
    n     & {tn} & {3.88060*lgn*sum} & {1.88234*sqrtn*sum} & {0.154339*n*sum} & {0.0230370*nlgn*sum} & {0.00193887*n\^2*sum} & {0.0000300791*n\^3*sum} & {8.76851E-18*2\^n*sum} & {2.31541E-80*n!*sum} \\
    \hline
    {20} & 4.030974 & 16.77167 & 8.41808 & 3.08678 & 1.991285 & 0.775548 & 0.240633 & 9.19E-12 & 5.63E-62 \\
    {21} & 3.863134 & 17.04483 & 8.625966 & 3.241119 & 2.124902 & 0.855042 & 0.278563 & 1.84E-11 & 1.18E-60 \\
    {22} & 3.65036 & 17.30527 & 8.828957 & 3.395458 & 2.260102 & 0.938413 & 0.320282 & 3.68E-11 & 2.59E-59 \\
    {23} & 3.830771 & 17.55413 & 9.027386 & 3.549797 & 2.396814 & 1.025662 & 0.365972 & 7.36E-11 & 6.00E-58 \\
    {24} & 4.05099 & 17.79241 & 9.221545 & 3.704136 & 2.534971 & 1.116789 & 0.415813 & 1.47E-10 & 1.44E-56 \\
    {25} & 4.226884 & 18.02095 & 9.4117 & 3.858475 & 2.674513 & 1.211794 & 0.469986 & 2.94E-10 & 3.59E-55 \\
    {26} & 4.663252 & 18.24053 & 9.598088 & 4.012814 & 2.815385 & 1.310676 & 0.52867 & 5.88E-10 & 9.33E-54 \\
    {27} & 5.068824 & 18.45182 & 9.780926 & 4.167153 & 2.957535 & 1.413436 & 0.592047 & 1.18E-09 & 2.52E-52 \\
    {28} & 5.266806 & 18.65542 & 9.960407 & 4.321492 & 3.100917 & 1.520074 & 0.660296 & 2.35E-09 & 7.06E-51 \\
    {29} & 5.16485 & 18.85188 & 10.13671 & 4.475831 & 3.245486 & 1.63059 & 0.733599 & 4.71E-09 & 2.05E-49 \\
    {30} & 5.007435 & 19.04168 & 10.31 & 4.63017 & 3.391201 & 1.744983 & 0.812136 & 9.42E-09 & 6.14E-48 \\
    {31} & 4.937482 & 19.22525 & 10.48043 & 4.784509 & 3.538024 & 1.863254 & 0.896086 & 1.88E-08 & 1.90E-46 \\
    {32} & 5.355774 & 19.403 & 10.64812 & 4.938848 & 3.68592 & 1.985403 & 0.985632 & 3.77E-08 & 6.09E-45 \\
    {33} & 5.485246 & 19.57528 & 10.81322 & 5.093187 & 3.834854 & 2.111429 & 1.080953 & 7.53E-08 & 2.01E-43 \\
    {34} & 5.485434 & 19.74241 & 10.97583 & 5.247526 & 3.984796 & 2.241334 & 1.182229 & 1.51E-07 & 6.83E-42 \\
    {35} & 5.906561 & 19.9047 & 11.13607 & 5.401865 & 4.135715 & 2.375116 & 1.289641 & 3.01E-07 & 2.38E-40 \\
    {36} & 6.290304 & 20.06241 & 11.29404 & 5.556204 & 4.287584 & 2.512776 & 1.40337 & 6.03E-07 & 8.61E-39 \\
    {37} & 6.096659 & 20.2158 & 11.44983 & 5.710543 & 4.440377 & 2.654313 & 1.523597 & 1.20E-06 & 3.20E-37 \\
    {38} & 6.182617 & 20.36511 & 11.60352 & 5.864882 & 4.594067 & 2.799728 & 1.6505 & 2.41E-06 & 1.21E-35 \\
    {39} & 6.20609 & 20.51053 & 11.75521 & 6.019221 & 4.748633 & 2.949021 & 1.784262 & 4.82E-06 & 4.72E-34 \\
    {40} & 6.280319 & 20.65227 & 11.90496 & 6.17356 & 4.90405 & 3.102192 & 1.925062 & 9.65E-06 & 1.89E-32 \\
    {41} & 6.473785 & 20.79052 & 12.05286 & 6.327899 & 5.060299 & 3.25924 & 2.073082 & 1.93E-05 & 7.76E-31 \\
    {42} & 6.819306 & 20.92543 & 12.19896 & 6.482238 & 5.217358 & 3.420167 & 2.2285 & 3.86E-05 & 3.26E-29 \\
    {43} & 7.083658 & 21.05716 & 12.34333 & 6.636577 & 5.375209 & 3.584971 & 2.391499 & 7.72E-05 & 1.40E-27 \\
    {44} & 7.121018 & 21.18587 & 12.48603 & 6.790916 & 5.533833 & 3.753652 & 2.562258 & 1.54E-04 & 6.16E-26 \\
    {45} & 7.061773 & 21.31169 & 12.62712 & 6.945255 & 5.693212 & 3.926212 & 2.740958 & 3.09E-04 & 2.78E-24 \\
    {46} & 7.325821 & 21.43473 & 12.76665 & 7.099594 & 5.85333 & 4.102649 & 2.927779 & 6.17E-04 & 1.27E-22 \\
    {47} & 7.459951 & 21.55514 & 12.90467 & 7.253933 & 6.01417 & 4.282964 & 3.122902 & 1.24E-03 & 6.00E-21 \\
    {48} & 7.588139 & 21.67301 & 13.04123 & 7.408272 & 6.175717 & 4.467156 & 3.326508 & 2.46E-03 & 2.87E-19 \\
    {49} & 8.05135 & 21.78844 & 13.17638 & 7.562611 & 6.337957 & 4.655227 & 3.538776 & 4.94E-03 & 1.41E-17 \\
    {50} & 9.170379 & 21.90155 & 13.31015 & 7.71695 & 6.500876 & 4.847175 & 3.759888 & 9.91E-03 & 7.04E-16 \\
    {51} & 7.931418 & 22.01241 & 13.4426 & 7.871289 & 6.664459 & 5.043001 & 3.990023 & 1.97E-02 & 3.59E-14 \\
    {52} & 7.960861 & 22.12113 & 13.57375 & 8.025628 & 6.828694 & 5.242704 & 4.229362 & 3.95E-02 & 1.87E-12 \\
    {53} & 8.090724 & 22.22777 & 13.70364 & 8.179967 & 6.993567 & 5.446286 & 4.478086 & 7.90E-02 & 9.89E-11 \\
    {54} & 9.425465 & 22.33242 & 13.83232 & 8.334306 & 7.159069 & 5.653745 & 4.736375 & 1.58E-01 & 5.35E-09 \\
    {55} & 8.944139 & 22.43514 & 13.95981 & 8.488645 & 7.325185 & 5.865082 & 5.00441 & 3.16E-01 & 2.94E-07 \\
    {56} & 8.962766 & 22.53602 & 14.08614 & 8.642984 & 7.491906 & 6.080296 & 5.282371 & 6.32E-01 & 1.65E-05 \\
    {57} & 10.00026 & 22.63511 & 14.21136 & 8.797323 & 7.65922 & 6.299389 & 5.570439 & 1.26E+00 & 9.38E-04 \\
    {58} & 9.966221 & 22.73248 & 14.33547 & 8.951662 & 7.827118 & 6.522359 & 5.868793 & 2.53E+00 & 5.44E-02 \\
    {59} & 9.630325 & 22.82818 & 14.45853 & 9.106001 & 7.995588 & 6.749206 & 6.177615 & 5.05E+00 & 3.22E+00 \\
    \hline
    MSE   &       & 189.819 & 26.6282 & 0.32812 & 2.89558 & 10.8037 & 18.0377 & 42.1988 & 4.49E+01 \\
    \hline
\end{tabular}}

Từ thực nghiệm nhận thấy độ phức tạp của thuật toán là : $ O(n) =n *sum$

\subsection{Bonus Advance case: Nếu việc tìm subset tạo thành Sum là không thể thì tìm kết quả gần sum nhất mà không vượt qua nó}
\subsubsection{Giải quyết bằng Backtracking}
\subsubsubsection{Mã giả}
\begin{framed}
\textbf{BackTracking}(X, i, sum, k, n):\\
\tab \textbf{Input:} k là tổng muốn đạt được\\
\tab \textbf{Input:} X là tập hợp được cho \\
\tab \textbf{Input:} n là Số phần tử của tập hợp X\\
\tab \textbf{Input:} i bước quay lui\\
\tab \textbf{Input:} sum tổng tạo được ở bước thứ i\\
\tab if sum > k : \\
\tab\tab return 0 \\
\tab else if i == n :\\
\tab\tab return sum\\
\tab return \textbf{Max}(\textbf{BackTracking}(i + 1, sum + set[i]), \textbf{BackTracking}(i + 1, sum))
\end{framed}
\subsubsubsection{Mã nguồn}
\begin{center}
    \begin{lstlisting}[language=Python, caption=Return nearest sum]
def backtracking(i, sum):
    if sum > k: 
        return 0
    if i == n:
        return sum
    pick = backtracking(i+1, sum + set[i])
    leave = backtracking(i+1, sum)
    return max(pick, leave)
\end{lstlisting}

\end{center}

\subsubsubsection{Thực nghiệm đánh giá độ phức tạp của case}
\textbf{Cấu hình máy thực nghiệm}
\begin{itemize}
  \item Chip: Intel Core i5 8250U
  \item RAM: 4GB
  \item Không card đồ hoạ
\end{itemize}
\textbf{Cách thức thực nghiệm}\\ \\
Ở mỗi lần gọi hàm \textit{backtracking} ta đều random 1 tập hợp $set$ với số phần tử n tăng dần từ 20 đến 59. Đặt mốc thời gian trước và sau khi gọi hàm \textit{backtracking} để tính time diff và ghi kết quả vào cột $tn$ tương ứng với mỗi $n$, quy trình này được thực hiện tự động bằng code.\\ \\
Tiếp tục tính $y = f(n)$ với $f(n)$ là các hàm độ phức tạp thông dụng đã biết. \\ \\
Sử dụng linear regrestion trên các giá trị tính được để tính hệ số X và MSE.\\ \\
Ta có kết quả thực nghiệm như bên dưới.\\ \\

\textbf{Kết quả thực nghiệm}\\

\scalebox{0.7}{
\begin{tabular}{|l|r|r|r|r|r|r|r|r|r|}
    \hline
    n     & {tn} & {64.0656*lgn} & {32.1249*sqrtn} & {2.71991*n} & {0.411456*nlgn} & {0.0362706*n\^2} & {0.000593185*n\^3} & {2.64556E-16*2\^n} & {8.37222E-79*n!} \\
    \hline
    {20} & 0.344618 & 276.8869 & 143.6669 & 54.3982 & 35.56566 & 14.50824 & 4.74548 & 2.77E-10 & 2.04E-60 \\
    {21} & 0.44056 & 281.3965 & 147.2148 & 57.11811 & 37.95215 & 15.99533 & 5.493486 & 5.55E-10 & 4.28E-59 \\
    {22} & 0.507078 & 285.6962 & 150.6791 & 59.83802 & 40.36692 & 17.55497 & 6.316234 & 1.11E-09 & 9.41E-58 \\
    {23} & 0.622521 & 289.8047 & 154.0656 & 62.55793 & 42.80867 & 19.18715 & 7.217282 & 2.22E-09 & 2.16E-56 \\
    {24} & 0.811223 & 293.7384 & 157.3792 & 65.27784 & 45.27625 & 20.89187 & 8.200189 & 4.44E-09 & 5.19E-55 \\
    {25} & 0.865186 & 297.5114 & 160.6245 & 67.99775 & 47.76856 & 22.66913 & 9.268516 & 8.88E-09 & 1.30E-53 \\
    {26} & 1.200351 & 301.1365 & 163.8055 & 70.71766 & 50.28463 & 24.51893 & 10.42582 & 1.78E-08 & 3.38E-52 \\
    {27} & 1.306201 & 304.6247 & 166.9259 & 73.43757 & 52.82353 & 26.44127 & 11.67566 & 3.55E-08 & 9.12E-51 \\
    {28} & 1.774107 & 307.9861 & 169.989 & 76.15748 & 55.38442 & 28.43615 & 13.0216 & 7.10E-08 & 2.55E-49 \\
    {29} & 2.054637 & 311.2295 & 172.9979 & 78.87739 & 57.96652 & 30.50357 & 14.46719 & 1.42E-07 & 7.40E-48 \\
    {30} & 2.334113 & 314.3629 & 175.9553 & 81.5973 & 60.56909 & 32.64354 & 16.016 & 2.84E-07 & 2.22E-46 \\
    {31} & 2.932169 & 317.3936 & 178.8639 & 84.31721 & 63.19145 & 34.85605 & 17.67157 & 5.68E-07 & 6.88E-45 \\
    {32} & 3.251418 & 320.328 & 181.7259 & 87.03712 & 65.83296 & 37.14109 & 19.43749 & 1.14E-06 & 2.20E-43 \\
    {33} & 3.378786 & 323.1721 & 184.5435 & 89.75703 & 68.49303 & 39.49868 & 21.31729 & 2.27E-06 & 7.27E-42 \\
    {34} & 4.857418 & 325.9314 & 187.3187 & 92.47694 & 71.17108 & 41.92881 & 23.31454 & 4.55E-06 & 2.47E-40 \\
    {35} & 4.795174 & 328.6106 & 190.0535 & 95.19685 & 73.8666 & 44.43149 & 25.43281 & 9.09E-06 & 8.65E-39 \\
    {36} & 6.298944 & 331.2143 & 192.7494 & 97.91676 & 76.57908 & 47.0067 & 27.67564 & 1.82E-05 & 3.11E-37 \\
    {37} & 7.103429 & 333.7468 & 195.4081 & 100.6367 & 79.30805 & 49.65445 & 30.0466 & 3.64E-05 & 1.15E-35 \\
    {38} & 7.723177 & 336.2116 & 198.0312 & 103.3566 & 82.05307 & 52.37475 & 32.54925 & 7.27E-05 & 4.38E-34 \\
    {39} & 10.15684 & 338.6125 & 200.6199 & 106.0765 & 84.81371 & 55.16758 & 35.18714 & 0.000145 & 1.71E-32 \\
    {40} & 11.99208 & 340.9525 & 203.1757 & 108.7964 & 87.58957 & 58.03296 & 37.96384 & 0.000291 & 6.83E-31 \\
    {41} & 14.57703 & 343.2348 & 205.6997 & 111.5163 & 90.38027 & 60.97088 & 40.8829 & 0.000582 & 2.80E-29 \\
    {42} & 14.24926 & 345.4621 & 208.1931 & 114.2362 & 93.18546 & 63.98134 & 43.94789 & 0.001164 & 1.18E-27 \\
    {43} & 15.63854 & 347.6369 & 210.6571 & 116.9561 & 96.00478 & 67.06434 & 47.16236 & 0.002327 & 5.06E-26 \\
    {44} & 19.0709 & 349.7618 & 213.0925 & 119.676 & 98.8379 & 70.21988 & 50.52987 & 0.004654 & 2.23E-24 \\
    {45} & 21.77789 & 351.8389 & 215.5004 & 122.396 & 101.6845 & 73.44797 & 54.05398 & 0.009308 & 1.00E-22 \\
    {46} & 26.01883 & 353.8703 & 217.8817 & 125.1159 & 104.5443 & 76.74859 & 57.73826 & 0.018616 & 4.61E-21 \\
    {47} & 32.00005 & 355.8581 & 220.2372 & 127.8358 & 107.417 & 80.12176 & 61.58625 & 0.037233 & 2.17E-19 \\
    {48} & 35.86215 & 357.804 & 222.5678 & 130.5557 & 110.3024 & 83.56746 & 65.60152 & 0.074466 & 1.04E-17 \\
    {49} & 36.50591 & 359.7098 & 224.8743 & 133.2756 & 113.2001 & 87.08571 & 69.78762 & 0.148932 & 5.09E-16 \\
    {50} & 53.18904 & 361.577 & 227.1573 & 135.9955 & 116.1099 & 90.6765 & 74.14813 & 0.297864 & 2.55E-14 \\
    {51} & 51.08501 & 363.4073 & 229.4177 & 138.7154 & 119.0316 & 94.33983 & 78.68658 & 0.595727 & 1.30E-12 \\
    {52} & 61.15088 & 365.2021 & 231.6559 & 141.4353 & 121.965 & 98.0757 & 83.40656 & 1.191454 & 6.75E-11 \\
    {53} & 79.61625 & 366.9627 & 233.8728 & 144.1552 & 124.9097 & 101.8841 & 88.3116 & 2.382909 & 3.58E-09 \\
    {54} & 75.44256 & 368.6903 & 236.0688 & 146.8751 & 127.8657 & 105.7651 & 93.40528 & 4.765817 & 1.93E-07 \\
    {55} & 77.25711 & 370.3863 & 238.2446 & 149.5951 & 130.8326 & 109.7186 & 98.69115 & 9.531634 & 1.06E-05 \\
    {56} & 89.28668 & 372.0517 & 240.4007 & 152.315 & 133.8104 & 113.7446 & 104.1728 & 19.06327 & 0.000595 \\
    {57} & 101.5724 & 373.6876 & 242.5377 & 155.0349 & 136.7987 & 117.8432 & 109.8537 & 38.12654 & 0.03393 \\
    {58} & 118.1911 & 375.2951 & 244.6559 & 157.7548 & 139.7975 & 122.0143 & 115.7375 & 76.25308 & 1.967942 \\
    {59} & 140.0269 & 376.8751 & 246.756 & 160.4747 & 142.8065 & 126.258 & 121.8277 & 152.5062 & 116.1086 \\
    \hline
    MSE   &       & 94782 & 29723.2 & 6581.42 & 3766.82 & 1301.28 & 425.758 & 141.24 & 1646.23 \\
    \hline
\end{tabular}}
\\
Từ thực nghiệm nhận thấy độ phức tạp của thuật toán là : $ O(n) = 2^n$
\newpage

\begin{thebibliography}{9}
\bibitem{latexcompanion} 
Subset Sum problem \\
\url{https://en.wikipedia.org/wiki/Subset_sum_problem}
\bibitem{latexcompanion} 
Subset Sum problem \\
\url{https://www.geeksforgeeks.org/subset-sum-problem-dp-25}
\bibitem{latexcompanion} 
Paper \\
\url{https://www.cs.dartmouth.edu/~deepc/Courses/S19/lecs/lec6.pdf}
\bibitem{latexcompanion} 
Closest of Target \\
\url{https://www.baeldung.com/cs/subset-of-numbers-closest-to-target}
\bibitem{latexcompanion} 
Dataset (seeder) \\
\url{https://people.sc.fsu.edu/~jburkardt/datasets/subset_sum/subset_sum.html}
\end{thebibliography}
\end{document}

